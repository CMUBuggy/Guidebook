\chapter{Chairman}
\label{ch:Chairman}

This chapter lists information specific to the role of Chairman. It should be
noted that anybody can perform these tasks, but they usually fall under the
scope of the Chairman's responsibilities.

\section{Orientation}
The first task of the Chairman within the year is the organization of Buggy
Orienation. This event is meant to introduce as many Freshmen as possible
to the sport of Buggy in hopes that they will join a team. The event is
typically held during Freshmen Orientation, although it should be noted
that direct recruitment is discouraged during this time period. For that
reason, this event should focus on Buggy as a whole and presenters should
not identify with any team if possible. Greek orgs rarely participate, but
every effort should be made to get as many teams to participate as possible.
\\
Slides for the presentation are located on Google Drive and can be easily
updated each year. The presentation is introduced by the Chairman, then
team members from each of the roles in Buggy (Pusher, Driver, Mechanic,
Support) present the basics of their roles and share stories about what
Buggy means to them. Teams should be encouraged to bring buggies to the event.
\\
After the presentation portion, there is typically an outdoor portion of the
event where Freshmen are invited for a closer look at buggies and Q\&A. There
is often an option for Freshmen to push a buggy. If there is a driver in the
buggy, it is extremely important that the driver be a veteran driver and be
wearing all required safety gear.
\\
To ensure maximum attendance, flyers should be put up as early as possible
before the event. If the Chairman will not be on campus during Orientation,
it is likely that one or more teams will be willing to help put up flyers on
their behalf. Example flyers can be found on Google Drive and can be modified
in Photoshop for the current year.

\section{Meetings}
\label{sec:Meetings}
One of the most important jobs of the Chairman is organizing and hosting
meetings. There are four commonly recurring meetings which the Chairman
is expected to host. It can be good practice to have the Assistant Chairman
take notes during meetings so that the Chairman doesn't have to multitask
running the meeting and jotting things down. Notes from all meetings can
be found on Google Drive in years 2015 and 2010, which provide a good
understanding of what to expect throughout the year.

\subsection{Chairmen's Meetings}
These meetings are described clearly in the rules. At the time of this
writing 10pm on Mondays has continued to work well as a meeting room. MM103
(Breed Hall) proves to be an ideal meeting location. It should be requested
as early as possible to avoid changing locations every week.
\\
Before rolls can even start, there should be at minimum one meeting of the
Chairmen to make sure, among other things, that everyone knows each other, that
everone is familiar with the rules (and that they are responsible for their
entire team's adherence to them), and that they understand what is expected
of them on a given day of rolls.
\\
It has proven a great idea to begin the first Chairmen's meeting with food
and socializing. The Chairmen are responsible for much of the decision making
throughout the year, and it will go much more smoothly if they know each
other and are comfortable speaking up in the meeting environment.

\subsection{1-on-1 Meetings}
These meetings are between the Chairman and the Sweepstakes Advisor and should
happen once-a-week whenever possible. It is amazing how much information
there is to share every week and how many questions can come up.
\\
From the Chairman's perspective, it is important to update the Advisor on
current events in the Buggy world. The Advisor has an enormous wealth of
historical knowledge, but often does not have time to stay updated on the
day-to-day workings of the Buggy community. The more the Advisor knows,
the more they can help the Chairman lead with confidence.
\\
This is also the time to ask any questions which do not directly relate to
any other members of the Sweepstakes committee. Anything which involves other
members should be handled in the Committee Meetings.

\subsection{Committee Meetings}
These meetings are attended by the Sweepstakes Committee and the Sweepstakes
Advisor and are for working out anything relating to Buggy in general. They
should be held on Thursdays or earlier in the week if possible so that the
Committee can effectively prepare for each upcoming weekend of Rolls.
\\
When a Buggy Book Chair and Buggy Showcase Chair are elected, they should be
invited to attend these meetings.

\subsection{Operations Meetings}
These meetings are only held in the leadup to Raceday. They typically begin
in January unless someone sees reason that they should start earlier. The
participants include at least one member from every organization that will
be helping to facilitate Raceday. This includes at minimum:
\begin{itemize}
\item Sweepstakes
\item cmuTV
\item WRCT
\item Radio Club
\item CMU EMS
\end{itemize}

It is typical for Alumni Relations and the Buggy Alumni Association to attend
these meetings as well, since an enormous number of alumni attend Raceday
every year. It is acceptable to only invite them later in the year, since
they are not as integral to the specifics of running Raceday, however they
often have knowledge or can offer assistance when needed.
\\
During these meetings, more than any other, it is important for the Chairman
to arrive with an agenda. Without it, it will be unclear who should speak and
take charge of the moment. That is the job of the Chairman. Specifically, cmuTV
and WRCT sometimes drag their feet with preparations, and need to be strongly
encouraged to prepare early. Like Sweepstakes, their leadership changes every
year, and they might not have a fancy guidebook to help them along.
\\
If a particular group does not appear in the
agenda, they do not have to attend the meeting, although often once
conversation starts flowing, it can be useful to have as many people
present as possible to weigh-in. The list of tasks to cover before Raceday
may seem daunting, but these people are all here to help make it a success.


\section{Signs}
\label{sec:Signs}
It may seem silly to have a section dedicated to signs, but there are a
surprisingly large number of signs involved in Buggy, and proper implementation
of these signs and their placement can make many jobs much easier for the
Chairman. Also, all except the last of these signs must be implemented before
Rolls can begin. Kinkos on campus is the most convenient place to go for
printing.

\subsection{Laminated No Parking Signs}
These signs are posted around the course and indicate to cars the dates when
they will be towed for parking on the course. They should say \"No Parking\"
and should include the variance number given by the city. One set is made for
Fall and one for Spring and they remain posted from at least one week prior
to Rolls beginning and remain posted all semester. PDF files of these signs
are available on Google Drive and can be edited in Photoshop to change the
dates each year.
\\
You should DEFINITELY have
Kinkos drill holes close to centerline for zip ties to pass through. This
saves the team whose chore it is to put them up the struggle of drilling holes,
and removes the possibilty that they will be poorly drilled or drilled in the
corners. Don't drill holes in the corners... The signs should
be at least 11''x14'' in size. Printing them on yellow paper also makes them
significantly more visible.
\\
In the past it has proved necessary to provide the team placing the signs
with a map of where they should be placed. Such a map is available on Google
Drive, but essentially consists of every signpost around the entire course
where a sign could possibly be posted. Sweepstakes also needs to provide
zip ties to the team placing the signs.
\\
A fresh set of laminated signs needs to be produced between Truck Weekend
and Raceday for the sole reason that Raceday does not occur on Saturday and
Sunday. You can prepare these ahead of time to remove one task during the
stressful time of Raceday.
\\
It would be nice if these signs could be made from metal and posted on metal
T-posts, however as of this writing the City of Pittsburgh forbids placing
T-posts in their ground.

\subsection{Weekly No Parking Signs}
In order to close roads, the City of Pittsburgh requires that paper No
Parking signs be posted no later than 24 hours prior to the beginning of the
closure. These signs are provided by the city (or copied on the Student
Activities copy machine) and need dates and variance numbers written in
manually. This task is historically carried out by the generous Sweepstakes
Advisor. The task of posting the signs is assigned to a team as
specified in the rules.
\\
Theoretically these signs are superfluous compared to the Laminated No
Parking Signs, but they should still be placed at visible locations around
the course. If it rains after they are posted, the signs will likely be
destroyed. It is probably fine not to worry about this except on Raceday,
when you should worry about EVERYTHING. In this case, it is very good practice
to print twice as many signs as necessary, just in case.

\subsection{Road Closed A-frames}
Large (22\"x28\" minimum) signs must be attached to A-frames and placed
at the entrance to every road that is closed during Rolls and Raceday. The
signs must be removed every day or they may be destroyed. They should be
placed in the middle of the road, not on the sidewalk. This is a chore for a
team to carry out, but it is Sweepstakes' responsibility to ensure that team
is doing it correctly.
\\
Typically the team whose chore it is to place the signs keeps them from
year to year. They will likely know how to do the job, however the Chairman
should drive around the course on the morning of every day of Rolls to ensure
the signs are placed properly and are clearly visible. During this check,
the Chairman should also check that the barricades are in place and the police
officers are attending them.

\subsection{Raceday Road Closed Foam-Core Signs}
These signs are for Raceday only. They must be printed on weatherproof
foam-core board (like posterboard but made out of plastic) to
ensure that they remain stiff even in rain. They should indicate in large letters
the dates and times when the roads will be closed during Raceday. As always,
editable examples are available on Google Drive. Also, make sure that the
printer places grommets near the centerline for the zip ties to go through.
Sweepstakes will need to purchase extra-large zip ties for the team who
will post the signs.
\\
The location and orientation of each sign is also very
important and not at all obvious, so a document with a map and photos is
available on Google Drive and should absolutely be given to the team who is
posting them.
\\
These signs must be posted a minimum of one week prior to Raceday.

\subsection{Barricade Signs}
These signs are optional, but have proved helpful and greatly increase
safety for drivers on the course. They are laminated yellow
signs which clearly indicate that bicyclists and pedestrians are not allowed
on the roads during Rolls and Raceday. With these, the barricaders have some
help with their jobs, since bikers and pedestrians often wear headphones and
cannot hear people screaming at them.
\\
If used, the signs can be attached to the barricades using tablecloth clips.
They can be purchased at REI and other locations. They can be augmented with
caution tape to block the entire road, making it impossible for bicycles to
easily breeze past a barricade.

\subsection{Crosswalk Signs}
These signs are optional. They are two signs which are posted on A-frames and
placed at either end of the crosswalk at the top of Hill 2. They indicate to
runners that they should wait for permission to cross, which greatly increases
the safety level for drivers in that area.


\section{Rolls Preparation}
\label{sec:ChairRollsPrep}
There are a number of things the Chairman needs to take care of before rolls
can even be scheduled.

\subsection{Haybales}
Hopefully you're aware that on each day of rolls the chute is lined with a
great many haybales. 300 to be exact, at least as of right now. However
those haybales don't appear from nowhere. The Chairman needs to order them
from a local farmer. At the time of writing, L.P. Campbell is the prefered
source. They know about Sweepstakes and are quick to respond. Currently there
are two phone numbers they might be reached at: (724) 899-2403, and
(412) 418-0623. They might not answer the first time you call. They're
farmers, cut them some slack.
\\
It is important, however, to call early. They will need about a week just
to load all the hay onto a truck, and they can only deliver on days when
it's not raining.

\subsubsection{Tarps}
Before the hay is delivered, Sweepstakes needs to purchase two large tarps.
30ftx60ft should be more than large enough.
\\
One of the biggest challenges you may face as a chairman is getting the tarps
not to blow away (unless you tell the team covering them to do it better).
I suggest purchasing sand bags to weigh down the top tarp. You can also use
cable clamps (reusable zip ties) to tie at least one edge of the top tarp
to the bottom one.
\\
At the end of each semester, go collect the tarps or they will disappear
forever.

\subsubsection{Delivery}
Receiving the haybales is pretty straightforward, however it is a terrible
chore and will give you empathy for the teams who do it every weekend.
Typically during delivery each team is expected to send at least one person to
help. It is perfectly reasonable to fine teams that do not send anyone.

\subsection{Barricades}
The City of Pittsburgh Department of Public Works is responsible for delivering
barricades for road closures. CMU is located in District 3. The barricades may
be delivered to the corner of Tech and Frew without any intervention by
Sweepstakes, however if Rolls are coming soon and there are no barricades,
a phone call should be made to the 3rd Division. Phone numbers should be
available on the Department of Public Works website.

\subsection{Road Closures}
Sweepstakes needs to request permission from the city to close the roads.
This task is usually handled by the Sweepstakes Advisor, however the Chairman
should verify that it is done in preparation for Rolls.
\\
Also, closures for Raceday should be requested as early as humanly possible.
You do not want someone stealing those days, trust me.


\section{Running Rolls}
\label{sec:ChairRunningRolls}
This section describes the duties of the Chairman on a typical day of Rolls.

\subsection{Police}
The police are given instructions to arrive at Tech and Frew at 5am on each
day of Rolls. The Chairman needs to meet them there and ensure that each one
knows what their job is. Some will be veterans, but others will have been
told nothing at all about their duty that day. There is a handout on Google
Drive with Sweepstakes' phone numbers and a map of the barricade locations
that can be given to new police.
\\
At this time it is very important to collect each of their phone numbers
and alert them to any unusual events happening on the roads. They MUST
be instructed to keep out all cars - even emergency vehicles = unless you
explicitly call and tell them to let them through. In the case where a driver
needs an ambulance, call all the police and tell them to stay at their
barricades but allow any ambulances through. You DO NOT want ambulances
coming through the course while buggies are rolling.
\\
Before the meeting, it is good practice to drive around the course and see
if there are any cars that need to be towed. At the meeting, tell any of
the officers where the cars are and they will have them tagged and towed.
Note that during this time, the barricade that they should be manning will be
unmanned, so the Chairman should ensure that the barricade is closed to
protect drivers during course walks.
\\
If only 3 police officers show up, don't send anyone to Circuit Dr.
If 2 or fewer police officers show up, you will need to elicit help from
the teams and get volunteers to man the outer barricades. In this case,
instruct them to park far back from the barricade so they don't get harrased
by angry drivers.
\\
When rolls are finished for the day, call each officer and tell them to open
their barricade. They should not assume that rolls are finished at 9:00,
in case there are still buggies on the course.

\subsection{Meetings}
\subsubsection{Barricaders}
Soon after the police have been sent to their posts, a meeting of the
barricaders should be called. The requirements for being a barricader are not
many (except for the Phipps and Margaret Morrison barricaders, who need to be
competent individuals. These are probably the most dangerous barricades).
Just ensure that each team has a barricader and that they know where to go
and to not let bikes or cars or runners onto the course. Make sure to impress
upon them that it is a safety issue and that they could be saving drivers'
lives.
\\
Each barricader needs to wear a safety vest for visibility. These can be
returned every day or kept all year. Either way, when they go missing,
the team should pay to replace them.
\\
You should also remind them to make sure their team switches them out
during rolls, especially on cold days.

\subsubsection{Nap Time}
After the Barricaders meeting is Chairman nap time. You can go hang out
with the Safety Chair while he/she does drops, or you can find the team
giving out hot chocolate and give them extra safety points. Or you can yell
at people to sweep harder and fix the haybales and generally be better at
their chores.

\subsubsection{Flaggers}
Each flagger should have flagged at least once before or have someone along
to teach them. Be sure to remind each flagger to watch for hazards downhill
and throw a stop flag if there are any dangers.
\\
Make sure that all stop flags are kept hidden from view except in actual
emergencies.

\subsubsection{Chairmen}
This meeting is held just before the start of Rolls. It is the time when
the Chairmen can voice concerns and Sweepstakes can impart any important
information about the day. It is usually quite short.

\subsubsection{Drive Around}
After the Chairmen's meeting it is very good practice for the Chairman to
drive around the outside of the course to ensure that all of the outer
barricades are in place and the police are having no issues. The trip
should conclude with a drive through the chute to ensure that the course is
clear and rolls are ready to begin. Radio Club also ensures that the course is
clear.

\subsection{Radio Club}
This section exists simply because no member of a Buggy team ever has
significant reason to know that Radio Club exists during rolls.
They are stationed at minimum: at the top of the hill, at the transition
flag, and in the chute. They will report to the Chairman when each team
has cleared the chute and the course is clear for the next team.

\subsection{EMS}
CMU EMS has a contract with Sweepstakes renewed each year to be on call
with one EMT in the chute during each day of Rolls. The EMT will be first
on the scene of any crash, and will determine in conjunction with the
Safety Chair when it is appropriate to call an ambulance if necessary.

\subsection{Team Communication}
The Chairman needs some form of communication with the teams during Rolls.
In recent years, the phone app Zello has been used successfully. It has
the benefit that no physical walkie-talkies need to be passed out and returned.
\\
The Chairman needs to announce the roll order, especially who is clear.
The roll order should also be written on a board at the top of the hill.
The board should be in the Sweepstakes office on the 3rd floor of the
University Center.

\subsection{Cancellation}
In the event that Rolls need to be cancelled due to weather or other terrible
circumstances, it must be done by 3am the morning of Rolls to avoid paying
for police officers. There is a special phone number to call to cancel the
police. The Sweepstakes Advisor should have it on file.
In addition, the following people need to be informed that Rolls are cancelled:
\begin{itemize}
\item Chairmen
\item CMU Police Department
\item CMU EMS
\item Radio Club
\item Sweepstakes Advisor
\end{itemize}

As with most decisions made as Chairman, once the decision has been made to
cancel, it is best to stick to it, even if some teams complain. The teams
elected you to be their leader, and they will respect your decision, no
matter what it is. What they will not respect is someone who can't make up
their mind. So if it's on the edge, make a choice and stick with it until
the bitter end, even if it means setting everything up just as it starts
pouring rain. The worst that happens is everyone gets a few more hours of
sleep.


\section{Finances}
\label{sec:Finances}
It is another job of the Chairman to monitor the finances of Sweepstakes
throughout the year. This comes in two parts: collecting from the teams, and
planning for next year.

\subsection{Fees}
Raceday Fees are described in some detail in the rules, so we don't need
to discuss them here.

\subsection{Fines}
It is a good idea to keep a spreadsheet of all the accumulated fines of each
team, so that you can refer back to it when you charge them. An imperfect
example can be found on Google Drive. Including dates with each fine is
a very good idea.
\\
Do not do what it says in the rules. Taking a deposit and requiring more if
fines accumulate is ridiculous. Just charge teams at the end of the year.
And change the rules if you get the chance.
\\
Half fines (e.g. for being late to meetings) are a great way to keep teams in
line (or at least to make a little extra money).
\\
Any time a team covers another team's chore, they are credited the amount
of the fine for the chore they covered. However, to ensure Sweepstakes does
not lose money, if, at the end of the year a team has a positive balance,
Sweepstakes will not pay them money, but they will instead receive a
congratulatory pat on the back.

\subsection{JFC}
The Joint Funding Committee oversees the distribution of the Student
Activities Fee to all recognized student organizations at CMU. Upon
viewing Sweepstakes' budget from previous years, you will immediately
realize that we get a hefty percentage of the funds JFC hands out. This
is largely because Buggy is recognized as one of CMU's longest-standing
traditions and is held in high regard. It is important to stay in good
graces and ensure an easy time for future Sweepstakes. This comes in two
parts.
\subsubsection{Spending}
Don't spend too much money. If it's not in the budget, that doesn't mean
you can't buy it, but ask yourself if it's really necessary.

\subsubsection{Planning}
Each Chairman is responsible for planning the budget for the next Chairman.
Don't make them hate you because of this... Always ask for more than you
really need and ALWAYS appeal if they decline to provide some things.

\subsection{BAA}
The Buggy Alumni Association, in addition to hosting a website, helping
out with Rolls and Raceday, and occasionally stirring up drama, also collects
donations. They probably have a sizeable pot of money that they won't
necessarily tell you about. However, if you have an idea to improve Buggy,
but it's not in the budget, BAA might be able to help.

\subsection{Special Allocations}
If you determine that Sweepstakes needs to purchase something that does
not appear in your budget for the year, that is exactly the time to ask
JFC for a special allocation. These are one-time funds for things that come
up unexpectedly during the year. In fact, asking for special allocations
throughout they year looks good during appeals at the end of the year,
because it shows that you tried to remain out of debt, even if you failed.

\section{SWAG}
SWAG stands for Stuff We All Get. Except not everyone gets it, and it's
usually not free.
\\
This section is mostly for fun, however I do HIGHLY recommend buying some
sort of SWAG for the Sweepstakes committee early in the year. It does
wonders for pulling everyone together and making them feel like part of a
team. If the SWAG you want has a minimum order of 6, order for your Advisor,
Buggy Book Chair, and Buggy Showcase Chair also. They'll appreciate it
when they materialize.
\\
This SWAG is not optional, and is mentioned again in the Raceday section.
Like all things Raceday, if you do it early, you won't have to worry about it
at the last minute:

\subsection{Raceday Shirts}
Everyone participating as \"Staff\" on Raceday needs a shirt to make them
easily identifiable. These shirts should be brightly colored. Long sleeves
are prefered to avoid sunburn in case of sunshine (haha).

\subsection{End-of-year Gifts}
It is certainly a nice gesture to get some small memorabilia for those that
helped you throughout the year. Once it's all over, you will know exactly
who I mean.
\\
If the year went particularly well and you're feeling generous, you may
also give gifts to the Chairmen.
\\
Theoretically Sweepstakes could pay for these gifts, however it is very
unlikely that there will be an excess of money in the account, and if there
is, next year's Sweepstakes would probably appreciate it more than the
Chairmen.
