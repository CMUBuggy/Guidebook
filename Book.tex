\chapter{Buggy Book Chair}
\label{ch:Buggy Book}
This chapter covers useful info about how to make an awesome Buggy Book
without much hassle.

\section{Design}
\subsection{Delegation}
Remember, first and foremost, that it is your job to ensure that the Buggy
Book gets made. It is not necessarily your job to create it. If you are not
an artist or a designer, find some friends who are and bribe them with cookies.
The more eyes on the project the less work each person has to do.

\subsection{Past Books}
Looking at past Buggy Books is a wonderful way to get inspiration for what you
want the design to be. They should all be online at cmubuggy.org. If they're
not, bug someone to put them up. And be sure to look further back than just
one or two years. Sometimes it takes a few years for layout or content to
drastically change.

\subsection{Photos}
There are many great photos on cmubuggy.org, but you might feel that these have
been overused in past books. Most people won't notice, so don't worry too
much, but if you start asking around, you'll find lots of people have photos
that aren't posted online. Ask specific teams for photos, or even go to
the library or Alumni Relations for historical photos.

\subsection{Heat Times}
There is this question every year about whether or not the Day 1 heats will
be in the Buggy Book. Since heat selection usually happens very close to
Raceday, it is sometimes not possible, and it will depend on your turnaround
time for printing. Don't make the book late just for that nice final touch.
\\\\
If you can't get the heat schedule in time, a good compromise is to put a
url or QR code in the book linking readers to a website which lists the heat
times. You can talk to the Chairman about possibly linking to an externally-viewable
heat schedule.

\section{Schedule}
Be aware of your deadlines. Start layout as early as possible. You
will not get information from the teams until the very last minute so make
sure to finish everything else so you can plop the information in quickly.

\subsection{Chairmen}
Set a deadline for the Chairmen well before you actually need the information.
Most teams will overshoot it by a few days, and some might lag for a week or
more. Press them hard to get at least a basic set of information, and be as
lenient as you can when they want to make updates later.

\subsection{Tales}
Tales don't take long to write, but they do take a long time to edit. If
nobody is stepping forward wishing to write the Driver, Pusher and Mechanic
tales, find some friends who do Buggy and bribe them with cookies to write a
tale.
\\\\
The Chairman, Ass Chair, Safety Chair, Showcase Chair, and even you yourself
will have to write a tale as well. Don't let them slack on this just because
they're special.

\subsection{Support Organizations}
Don't forget about these guys! They will usually be very cooperative and
timely as long as you give them enough notice to put a page together.
Raceday would be nothing without them so they deserve a mention.

\subsection{Printing}
Research printers early and decide which one you want to go with. Figure out
exactly what the turnaround time is for printing. You will need every day you
can spare before the book has to go to print, but you also want to keep the
cost down.
\\\\
Remember that you will also need time to make a proof for both you and the
Sweepstakes Advisor to review. This will take at least one full day. And
give yourself time just to read through the entire book cover to cover
and make sure you don't have typos or other glaring mistakes.
