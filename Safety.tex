\chapter{Safety Chair}
\label{ch:Safety}

This chapter includes all information that past Safety Chairs felt necessary
to pass on. Read at your own discression.

\section{Drivers' Meeting}
\label{sec:Driver's Meeting}
One of the first things you will have to schedule and host is the Drivers'
Meeting. Typically two meeting times are scheduled to ensure that a majority
of drivers can make it to at least one. You will have to meet separately
with anyone who can't make either.
\\\\
In some years alumni have been invited to speak at these meetings to impress
upon the drivers the seriousness of their role. It is up to you whether
or not to take them up on this offer.
\\\\
Be sure to cover all aspects of Buggy related to safety, especially safety
gear and stop flags. Discuss how a properly fitted helmet should feel, and how
to form a mouthguard and cut off the ends so it doesn't choke you. Describe
Capes and a typical day of Rolls. Tell them anything you would want the
drivers on your old team to know. And make absolutely sure to impress upon
them that if they feel unsafe, they should stop, even on Raceday.
\\\\
It is also a good idea to show a video of the course
from a driver's perspective. If you don't have one of these, ask around,
they exist.
\\\\
You may consider giving the drivers a short quiz at the end to make sure they
were paying attention. This might seem silly, but even veteran drivers might
assume they know something when they actually don't. Some example questions
include:
\begin{itemize}
\item What are the 5 pieces of safety gear you are required to wear at all
times while driving?
\item What does a stop flag look like? What should you do when you see one?
\item How many points of attachment should your harness have?
\item If your helmet slips over your eyes, what should you do?
\end{itemize}

It's also a good idea to provide a handout summarizing what was discussed.
This can also help keep you on track during the meeting. And afterwards,
make sure to open the floor to Q\&A. The drivers should be comfortable with
you and feel that they can come to you with any problems they can't discuss
with their team.

\section{Safeties}
\label{sec:Safeties}
Some teams will try to make your life difficult, or will try to hide things
from you during safeties. Teams are allowed to have secrets, but if they
pertain to the safety of the driver, it is your job to make sure that whatever
is inside the buggy is safe for the driver.
\\\\
There's no reason to be overly lenient with the teams. They know the rules,
and they should be expected to follow them. That said, demanding unreasonable
things beyond the rules is not something you should regularly engage in.

\section{Capes}
\label{sec:Capes}
Teams will always schedule Capes at the leat convenient possible time.
Sometimes this will the morning of Rolls. You must maintain a balance
between doing your job and maintaining your sanity. If needed, you can
always ask the Chairman or Ass Chair to do Capes for you, assuming the
team is OK with it.
\\\\
If a team fails many times in a row, even if it's fewer than the number
specified in the rules, you can just tell them to try again later. Unless
they're super close, they obviously need to go back to the shop and fix
things anyway before it will be really safe.

\section{Spot Safeties}
\label{sec:Spot Safeties}
During each day of Rolls you should go and spot safety at least one team.
Work this out with your Chairman and Ass Chair so someone is available to
drive down to the chute in the event of a crash while you're busy.
\\\\
Don't give too many passes on Spot Safety failures. If there's a
problem that you told them about during Safeties or Capes and they haven't
fixed it by Rolls, a Spot Safety failure is the next level of warning. Or if
a driver clearly isn't harnessed in properly during the pull test, a Spot
Safety failure is a great way to make sure that team never makes that mistake
again. That said, it's still a pretty harsh punishment, so don't abuse it.

\section{Rule Changes}
If you see a place in the rules where safety could obviously be improved,
or if something comes up during the course of the year that is clearly
unsafe and needs to be banned, change the rules! You are in a unique
position for this year only where you have the power to do that. Make good
use of that opportunity!
\\\\
At the same time, realize that the rules have been around for many decades,
and most of the rules are there for good reasons, so proceed with caution.

\section{Safety Points}
\label{sec:Safety Points}
You can award teams safety points when they do safe things and be just like
Jake Reid. But really a more reasonable thing is to dole out punishment when
teams are unsafe.
