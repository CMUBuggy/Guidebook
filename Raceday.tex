\chapter{Raceday}
\label{ch:Raceday}
There are relatively few coherent memories of Raceday, but this chapter
aims to sum up all remembered information not covered in the Rules surrounding
the planning and execution of Raceday.

\section{Preparation}
Raceday is an amazing collaboration of an enormous number of people. But
don't let that freak you out. Most of those people are there to help in any
way they can.

\subsection{Sunday Raceday}
It is very important to plan for the possibility of Raceday happening on
Sunday. Though this has not happened in many years, it has happened, and it
is always possible given the Pittsburgh weather. The most important thing
is to let everyone involved know that it's a possibility. However, some
things just won't be able to happen. For example, it is unlikely that you
will be able to reserve the Jumbotrons on a potential basis for that date.
\\\\
Bottom line: if Raceday gets rained out on both Friday and Saturday, Sunday
will be only the basics. Make sure you have a rain date set up for:
\begin{itemize}
\item Road Closures
\item Police
\item Pittsburgh EMS
\item Timing System
\item Lead and Follow Trucks
\item Volunteers
\end{itemize}
It is assumed that cmuTV, WRCT, CMU EMS, and Radio Club will be available
that day, but double check with them.
\\\\
Discuss this with the teams as well. Booth teardown happens that day and
teams would be hard-pressed to fit in both things. It's good to talk to
the Carnival Committee Chair and ask if teams could get an extension on the
teardown deadline in that situation.

\subsection{Lead/Follow Trucks}
At some point before Raceday -- probably just a few days before --
you will need to pick up the Lead and Follow Trucks and bring them to
campus. This is pretty fun. Just make sure you plan ahead, because you will
need three drivers total -- two for the trucks, and one to get the drivers
there.

\subsection{Awards}
You will need to get the trophies made well in advance of Raceday. You will
also need to collect the first place ``travelling'' trophies from the last
year's winners so that their team name can be engraved. Note: the travelling
trophies will not have the current year's winners engraved on them. Instead
the winners will receive plates in the year following their victory.
\\\\
Determine the number of people that will be allowed on stage for awards and
make this clear to the teams. You want to limit it because the stage
probably won't be terribly large.

\subsection{Rosters}
Make sure to get rosters in a searchable format like an Excel document.
For preliminary rosters especially, this is incredibly important.

\subsection{Volunteers}
You will need to train your volunteers in how to judge whether a pusher
violated a transition. Get a team (usually Fringe) to bring out a buggy for
about a half hour and demonstrate to the volunteers what a good transition
and a bad transition looks like.
\\\\
Make sure that the line judges are properly recording any violations. It is up
to you how much you want to trust the judges to make the call. If you don't
trust them, have them film everything and review the tapes later. If you
trust them, then let them write down violations even if they didn't catch
them on film.
\\\\
Related to this, you will need to give each judge a handout with each heat,
the buggies in the heat, and identifying marks for each buggy so that someone
with no knowledge of the teams could accurately write down comments about what
they see.
\\\\
You also need to make bright shirts for everyone. The Chairman should have
this covered if they read the Chairman chapter.

\subsection{Announcers}
WRCT will need at least three announcers during each day of Raceday. This
is technically their job, but they usually don't know who would make good
announcers. There is a short list of alumni who make good announcers and you
should contact them and ask if they would like to do it again. Possibilities
include:
\begin{itemize}
\item Andy Bordick
\item Will Weiner
\item Connor Hayes
\end{itemize}

Try to figure out a good way to keep the announcers up to date on what's
happening on Raceday. You don't want them announcing the wrong thing.

\subsection{Supplies}
You can make sure you have all the supplies you need well in advance
so that you're not even more stressed out just before Raceday. Things
you will definitely need:
\begin{itemize}
\item Water bottles
\item Clipboards - one for each judge/timer
\item Pens - one for each clipboard
\item Digital Cameras (with working batteries) - one for each judge
\item Stopwatches (with working batteries) - one for each timer
\item Stanchions - for around the finish truss
\item Breakfast
\item Scissors
\end{itemize}

\subsection{Fire Safety Training}
Work with the Fire Marshal early in the Spring semester to go over how the
fire safety training will go and what will be discussed. There are certain
historical problems that should definitely be covered.

\subsection{Course Marshals}
You will need to work out a map of where the course marshals from each team
are to be stationed. By now you should know which teams are more reliable
than others, so put people you trust in important areas like the chute and
the top of the hill.
\\\\
You can use Radio Club to check up on the status of the Course Marshals
throughout the day to make sure nobody is slacking off.

\subsection{Timing System}
\subsubsection{Countdown Clock}
I recommend requesting a countdown clock for the starting line from the
timing guys. This is helpful for the teams to know when the next heat is.
But make sure they know that the starter's countdown from 10 is the official
start of the race. This is easiest when you tell the starter to count down
from 10 AFTER the countdown clock reaches 0.

\subsubsection{Stage/Tent}
You will need to order a stage and tent for the timing guys to set up on at
the finish line. This used to be a truck, but a stage and tent is much easier
to manage.

\subsection{Barricade Slips}
You will need to make and print barricade slips to give to anyone who might
need to come through the outer barricades. The Chairman needs to personally
sign each one with metallic sharpie. Do this ahead of time!


\section{Day-of}
\subsection{Stretch}
The number-one piece of advise for Raceday is stretch your legs and wear
good shoes. Especially if you're riding in the truck, jumping in and out all
day is really hard on your legs.

\subsection{Decisions}
It is very important for every member of Sweepstakes to realize that all
decisions made during Raceday are ostensibly final. Either there will be
no time to reverse the decision, or doing so would cause undue stress for
everyone.
\\\\
Think hard about each decision before you make it, and consult the teams'
Chairmen if necessary, but once you make a decision, stick to it. There is no
situation in which your decision will please 100\% of people, but you will
displease a lot more people if you go back on a decision after making it.
\\\\
If you discover that your decision truly was the wrong one, correct it,
and do so as quickly and openly as possible. If the decision is related to
the weather, and you've waited as long as reasonably possible to make the call,
don't worry about what anyone says. You're just doing your job.
\\\\
Also, as much as possible stick to the rules. Nobody can get mad at you
if you follow them to the letter.
\\\\
And don't forget about tape review. On Day 1, you can review the tapes at the
end of the day and grant re-rolls on Day 2. During the finals you need to
review the tapes immediately, but it can definitely aid in decision making.

\subsection{Protests and Appeals}
Make it clear ahead of time what teams can and cannot appeal. It takes a
lot of work to review each appeal and they should really have a limit
to how many they can submit.

\subsection{Disqualifications}
Organize a system for recording DQ's ahead of time. This is the
job of the Head Judge, but make sure that they hand off this information
to Sweepstakes at the end of the day! Otherwise nobody will know what the
results were during Chairmen's.

\subsection{Radio Shadows}
Nobody tells you about this because nobody except Sweepstakes knows, but
every member of Sweepstakes will have a Radio Shadow. This is someone from
Radio Club who will follow you around literally wherever you go (except in the
trucks for safeties or into the bathroom). They'll even run if you have to
be somewhere quickly! This is a pretty incredible resource because you have
instant communication access to everyone around the entire course.

\subsection{Starting Beep}
Triple check that this is working. It is so often a problem, just make sure
it's set up and ready to go. Then check it again during the timing heat.

\subsection{Tape Review}
Some of the tapes may need to be reviewed immediately by the Head Judge if
there is a protest or appeal, but the rest of the footage should be reviewed
at the end of the day in the cmuTV truck. You should review the tapes for
any questionable DQs to verify them, and review the camcorder footage from the
line judges for any heats where they made notes.
\\\\
It is not worth your time to pour over the entire day's worth of footage
looking for things someone might have missed. Be happy with the calls that
were made and accept them as final. It is also not necessary to involve the
Chairmen in this review. They will only slow down the process and argue about
trivial things. Once again, they elected you to make decisions.

\section{Chairman}
\subsection{Live Spreadsheet}
USE A LIVE SPREADSHEET!!! I can't tell you how strongly I recommend this.
\\\\
What is a live spreadsheet? It's like a Google Docs spreadsheet, except it's
published to a special website. Regular Google Docs spreadsheets slow down
or become impossible to edit once a certain number of people are viewing
them, but live spreadsheets don't have this problem.
\\\\
The trick is that the special website is updated every 30-60 seconds from
an actual Google Docs spreadsheet that you can edit during Raceday! This is
unbelievably useful when schedule changes inevitably happen. Even if the
next roll is delayed by only 2 minutes, you can update the spreadsheet and
everyone around the entire course will be notified within 1 minute.
\\\\
So how exactly does it work? First you need to make a spreadsheet in Google
Docs containing the schedule for Raceday. It is to your benefit to make this
spreadsheet as sophisticated as possible. You should include variables for
the time in between heats in case you need to shorten this time due to rain.
You should also link all heat times to the previous heat time, so that if you
change one time, the change propagates through all the heats with only one
step. There is a (used) example in the 2015 Google Docs folder.
\\\\
It is extremely important that you do NOT send out a link to the editable
spreadsheet, or if you do, make sure that those people do not share it and
do not open it on Raceday. As mentioned above, if too many people view this
document you will be unable to edit it.
\\\\
Once you have your spreadsheet set up, click File --$>$publish to the web. This
will give you a link to the live spreadsheet, which you can share with any
number of people. You can do this for just the sheet you want people to see
during Raceday. You can even publish to the web again on Day 2 with a different
sheet so people don't even have to change their link.
\\\\
The final step for making this successful is having a smart phone with 3G and
access to Google Drive. This is how you will make live updates.
It might seem like there is no time for this, but you'll be sitting in the
back of a truck for at least two minutes during every heat, which should be
plenty of time to update the schedule.
\\\\
Now you can communicate with all of the volunteers and teams in a single step.
You should have a backup method prepared, and make sure your Advisor has the
phone numbers of all the volunteers, but if you test out this system ahead of
time you shouldn't have any problems except for extra strange things like
weather advisories, which is what the alert system is for.

\subsection{Lead Truck}
As Chairman you get to ride in the Lead Truck along with the Head Judge,
your radio shadow, and possibly someone from the lead truck auction.
This is truly one of the coolest experiences you will ever have. Use this time
to take some deep breaths and realize what an amazing thing you've just helped
create.
\\\\
Be sure to wear comfortable shoes and stretch your legs before Raceday.
At the end of every heat you will jump out of the back of the truck and
observe one of the buggies crossing the finish line to ensure the pusher had
their hand on the bar.

\subsection{Safeties}
If you decide with your Head Judge and Safety Chair that you would like to
safety all three buggies after they race, then on Day 1 the Chairman will need
to do safeties as well. Just coordinate this properly and don't safety your own
team.

\section{Assistant Chair}
As the Assistant Chair, your job will be to man the starting line while
the Chairman and Safety Chair are running around the course like chickens
with their heads cut off. You will be the main point of contact for the teams
while the Chairman is away. You will also receive all protest and appeals
forms. Be sure to give these to the Head Judge for review. Only involve the
Chairman and Safety Chair if necessary, since they'll have plenty of things on
their minds.

\subsection{Course Walks}
You will be in charge of course walks in the morning. Drivers need to
walk with their heats and check in with you before they leave and after they
get back. Bring a list and check them off.

\subsection{Chair}
Bring a chair.


\section{Safety Chair}
\subsection{Truck Inspections}
Hopefully you've worked out most of the kinks during Fire Safety Training
and Truck Weekend. If you issued a warning on Truck Weekend that wasn't
fixed by Raceday, don't be afraid to disqualify people.
\\\\
Don't let teams obviously hide things. You shouldn't hesitate to ask them to
uncover something for the sake of safety.
\\\\
Be sure to check that any tarps covering the entryways are fireproof.

\subsection{Rain Delay}
If it rained earlier in the day and the roads are still damp, it will be
largely up to you to determine if it is safe enough to race. Taking drivers
down into the chute to inspect the roads is not a bad idea. Consulting the
Chairmen is also necessary, as ultimately it will need to be a joint decision.
However if everyone says ``GO'' and you still believe it is unsafe, it is your
duty more than anyone to say no.

\subsection{Follow Truck}
You will ride in the Follow Truck for every heat along with your radio shadow
and one representative from every team participating in that heat.
Be sure to check that everyone has an extraction kit. If they don't that is
perfectly obvious grounds for a DQ. Though of course you should warn them about
this and every other way they could possibly get disqualified.
\\\\
In the event of an accident you will be quickly on the scene.
\\\\
Wear comfortable shoes and stretch your legs. You will need to jump out of the
truck after every heat and run to safety the buggies from that heat.

\subsection{Ways To Get Disqualified}
You should dedicate a Chairmen's meeting prior to Raceday to going over all
of the ways a buggy or team could be disqualified. All of these are in the
rules, but there is no comprehensive list, so we will attempt to make one
here. Please update it if you think of more, and remember that teams will
always find new and interesting ways to disqualify themselves.
\begin{itemize}
\item Transition violations
\item No hand on the bar at finish
\item Pacing
\item Interfering with another team's buggy
\item False start (third time)
\item Unsportsmanlike conduct
\item Failing or interfering with drops
\item Failing or interfering with safeties
\item 5-second rule
\item Loss of mass
\item No one in follow truck with extraction kit
\item Interfering with EMS access to an accident
\item Fire safety violation

\end{itemize}

\section{Head Judge}
\label{sec:HeadJudge}
Hopefully you already have some idea of how Raceday works, unless you
completely blocked it from your memory. Still, some tips are always nice.

\subsection{Read The Rules}
This should be obvious, but review the rules ahead of time! Especially the
ones that can get people DQ'd. They are your best friend when making calls
that might be perceived as being on the borderline. And as always, when you
make a decision, stick to it.

\subsection{Protest and Appeals Forms}
As with everything, there are forms on Google Drive designed to record all
protests and appeals. Print these out before Raceday so that Sweepstakes
doesn't have to worry about it. Give the entire pile to the Assistant Chair,
you won't be around to collect them from the teams.
\\\\
If the form requests a reroll, and it's the second day of races, you need
to rule on it before the end of the day. If tape review is required, you can
use Radio Club to inform cmuTV of what you want to review so they can have it
ready. Hopefully they've figured out how to have a review booth at the top
of the hill, otherwise you'll need to run to the cmuTV truck and back.
\\\\
Most other protests can be reviewed at the end of the day. But don't wait too
long or else cmuTV might shut down operations and replays will be impossible.

\subsection{DQs}
Protest and Appeal Forms are a good place to record DQs throughout the day,
however you MUST decide on a way to pass this info to Sweepstakes when the
day is done. You can hand them the pile of forms or you can compile the data
into a spreadsheet, but make sure they have it! Or else go to Chairmen's
and announce it.

\subsection{Chairmen's}
At your discresion you may choose to attend the post-Raceday Chairmen's
meeting. Depending on how many teams you DQ'd you may not want to do this.
By not showing up, you give the teams less room to argue with decisions
that are supposed to be final. But if you really want to go defend your choices
in person, have fun!

\subsection{Help Out}
You know how hard it is running the show. Show up as early as you find sane
and help out in what ways you can think of. You know, if you're a nice person.
